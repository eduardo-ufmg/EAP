\section{Higher dimensional data}

\subsection{Limited visualisation}

\begin{frame}[allowframebreaks]{Higher dimensional data}
  \begin{table}[!htbp]
    \centering
    \caption{Four-dimensional train samples}
    \csvautotabular{../results/4dtrain.csv}
  \end{table}

  \begin{table}[!htbp]
    \centering
    \caption{Four-dimensional test sample}
    \csvautotabular{../results/4dtest.csv}
  \end{table}

  \note{Can't properly visualize space <br>
  Still have the numbers <br>
  What can we compute?}

\end{frame}

\subsection{Distance}

\begin{frame}{Distance}
  \begin{table}[!htbp]
    \centering
    \caption{Four-dimensional train samples with distances}
    \csvautotabular{../results/4dtraindist.csv}
  \end{table}

  \note{How can we use those distances to predict the test point?}

\end{frame}

\subsection{Closeness}

\begin{frame}{Closeness ranking}
  \begin{table}[!htbp]
    \centering
    \caption{Four-dimensional train samples ranked by distances}
    \csvautotabular{../results/4dtrainsorted.csv}
  \end{table}

  \note{ }

\end{frame}

\begin{frame}{Nearest neighbor label}
  \begin{table}[!htbp]
    \centering
    \caption{Four-dimensional test sample labelled by its nearest neighbor}
    \csvautotabular{../results/4dpred.csv}
  \end{table}

  \note{ }

\end{frame}
