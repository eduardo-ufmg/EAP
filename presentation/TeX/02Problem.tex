\section{The classification problem}

\subsection{Classifying rodents}

\begin{frame}{Problem introduction}
  \begin{itemize}
    \item Two species
    \item Count sightings of each
    \item Take some measurements
  \end{itemize}

  \note{Watching a nature reserve <br>
  There are two species of rodents <br>
  Count sightings of each species <br>
  Take some measurements}

\end{frame}

\begin{frame}{Training data}
  \begin{itemize}
    \item Species are distinguishable by fur color
    \item Measure body length and width with a camera
  \end{itemize}

  \note{Species are distinguishable by fur color <br>
  Measure body length and width with a camera}

\end{frame}

\subsection{Labelled data}

\begin{frame}{Day measurements}
  \begin{figure}[!htbp]
    \centering
    \small{\caption{Day sightings plot}}
    \includegraphics[scale=0.5]{../results/2dday.png}
  \end{figure}

  \note{Able to distinguish between species <br>
  Measurements are plotted}

\end{frame}

\subsection{Unlabelled data}

\begin{frame}{Night measurements}
  \begin{figure}[!htbp]
    \centering
    \small{\caption{Night sightings plot}}
    \includegraphics[scale=0.5]{../results/2dnight.png}
  \end{figure}

  \note{It is dark at night <br>
  Unable to distinguish between species <br>
  Measurements are plotted}

\end{frame}

\begin{frame}{Data superposition}
  \begin{figure}[!htbp]
    \centering
    \small{\caption{Superimposed sightings plot}}
    \includegraphics[scale=0.5]{../results/2dtraintest.png}
  \end{figure}

  \note{Patterns are similar <br>
  Species may be distinguishable by the measurements <br>
  How would you label this sample? <br>
  There are harder regions}

\end{frame}
